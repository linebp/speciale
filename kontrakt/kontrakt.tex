
\documentclass[10pt, a4paper]{article}
% babel: Hyphenation patterns and language specific strings
%\usepackage{babel} 
% isolatin1: Danish character set
%\usepackage{isolatin1}
%\usepackage[T1]{fontenc}
%\usepackage[latin1]{inputenc}
% varioref: Smart references
\usepackage{varioref}
% fancyheadings: Fancy headers
\usepackage{fancyhdr}
% avantgar, chancery, charter, courier, helvetic, newcent, pandora,
% palatino, times, utopia, zapfchan: Fonte
% euler: Math font
\usepackage{euler}
\usepackage{palatino}
%\usepackage{newcent}
% listings: For listing code with syntax highlighting
\usepackage{listings}
% graphicx: Include graphic-files
\usepackage{graphicx}
% amsmath: more mathcommands
\usepackage{amsmath}
% a4wide: for longer lines
\usepackage{a4wide}
% Package for infix arithmetic notation
\usepackage{calc}
% Defines if/then/else macros for conditional text
\usepackage{ifthen}
% Place selected parts of a document in landscape
\usepackage{verbatim}
\usepackage{url}

\usepackage{booktabs}
\usepackage{multirow}
\usepackage{rotating}

%\usepackage{bibmods}
%\usepackage{showtags}

%------------------------------------------------------------
% Set up listings environment
% /usr/local/stow/share/texmf/doc/latex/styles/listings.dvi
%------------------------------------------------------------
%\lstloadlanguages{}
\makeatletter
\lstset{
  language=C,
  columns=fullflexible,
  basicstyle=\ttfamily\small,
  keywordstyle=\bfseries,
  showstringspaces=f,
  breaklines=true,
  keepspaces=true
}
\lstset{
  numbers=left, 
  numberstyle=\tiny, 
  stepnumber=2, 
  numbersep=5pt, 
}
\makeatother

%------------------------------------------------------------
% Set up headers and footers
%------------------------------------------------------------
\pagestyle{fancy}
\fancyhead{}
% Put pagenumber centered on bottom of page
\lfoot{}
\cfoot{\thepage}
\rfoot{}
% Put section number and name on top left of page
\renewcommand{\sectionmark}[1]{\markboth{\thesection\ #1}{}}
\lhead{\fancyplain{}{\leftmark}}
% Make sure the rest is blank
\chead{}\rhead{}


%% %------------------------------------------------------------
%% % My own titlepagedefinition
%% %------------------------------------------------------------
%% \makeatletter
%% \renewcommand{\maketitle}{
%%   \begin{titlepage} 
%%  \let\footnotesize\small
%%     \let\footnoterule\relax
%%     \parindent \z@
%%     \reset@font
%%     \null
%%     \vskip 50\p@
%%     \begin{center}
%%       \hrule height 1pt
%%       \vskip 1pt 
%%       \hrule
%%       \vskip 3pt
%%       {\huge \bf \strut \@title \strut}\par
%%       \vskip 1pt
%%       \hrule
%%       \vskip 1pt
%%       \hrule height 1pt
%%     \end{center}
%%     \vskip 50\p@
%%     \begin{flushright}
%%       \Large \@author \par
%%       \vskip 50\p@
%%       \sl \@date
%%     \end{flushright}
%%     \vfil
%%     \null
%%   \end{titlepage}
%% }
%% \makeatother


%------------------------------------------------------------
% My own lists
%------------------------------------------------------------
\newlength{\Mylen}
\newcommand{\entrylabel}[1]{%
  \settowidth{\Mylen}{#1}%
  \ifthenelse{\lengthtest{\Mylen > \labelwidth}}%
  {\parbox[b]{\labelwidth}%
    {\makebox[0pt][l]{#1}\\}}%
  {#1}%
  \hfil\relax}
\newenvironment{entry}
{\begin{list}{}%
    {\renewcommand{\makelabel}{\entrylabel}%
      \setlength{\labelwidth}{35pt}%
      \setlength{\leftmargin}{\labelwidth+\labelsep}%
    }%
  }%
{\end{list}}

\title{Regular expression libraries and tools and applications}
\author{Line Bie Pedersen 300976}
%\date{}

\begin{document}
\begin{entry}
\item[\bf{Project title:}] A streaming full regular expression parser
\item[\bf{Author:}] Line Bie Pedersen (300976)
\item[\bf{Project consultants:}] Fritz Henglein and Lasse Nielsen
\item[\bf{Type:}] Master's Thesis, 30 ECTS
\end{entry}

\section*{Background}

%% omstrukturer. det er for løst
%% hvad er problemet i det oprindelige arbejde, hvad er det vi løser

In 2000 Dub\'{e} and Feeley \cite{Dube2000} published an article on
how to efficiently build a parse tree from a regular expression. To do
this they add output to the edges of the NFA. The problem with the
method described by Dub\'{e} and Feeley is memory consumption. The
method uses memory linear in the size of the input. We wish to extend
their theory such that memory consumption will lessen, using constant
memory where possible.

Henglein and Nielsens article from 2010 \cite{Henglein2010} describes
how parse trees can be compacted as bit-values. Bit-values are strings
of bits, indicating the choices taken when traversing the NFA. These
bit-values, combined with the original NFA, can for example be used to
determine if a string matches a regular expression.

Another important use of the bit-values is to report the contents of
groupings. There are different degrees of reporting contents of
groupings. The usual method, used by tools such as Perl and RE2,
reports only one match per set of parenthesis; This is called a
regular expression matcher. If a grouping is repeated by e.g. a star,
thus creating several matches per set of parenthesis, only a so called
\textbf{full parser} will return all these matches. A bit-value will
enable the return of all matches. A third option for regular
expression matching is a language acceptor, simply stating a yes or a
no depending on whether it matched or not.

%Language acceptor: Ja/nej 
%regular expression matcher: perl

%% These idea are used in this project; The hope is that by using a more
%% compact representation of the parse tree, memory consumption will
%% lessen.

%% This idea is used in this project to construct bitstrings representing
%% the choices made traversing the NFA with a textstring. These
%% bitstrings, combined with the orginal NFA, can for example be used to
%% determine if the string matches the regular expression, how it matches
%% and also find and rapport any groupings.

By making the functions streaming\footnote{A function is
  \textbf{streaming}, if it uses a constant amount of working memory.
  Working memory is all memory used, not counting memory used for
  input and output.}  and using the bit-strings to communicate between
the different functions, the hope is that memory consumption will
lessen.



%explain online!


%explain full!


\section*{Objectives}
The main objective of this project is to implement a streaming regular
expression full parser and evaluate and compare for example memory
consumption and run-time usage. The overall structure of the
communicating processes will be:
%memory consumption, runtime

\begin{equation*}
  \text{NFA simulator}\rightarrow\text{filters}\rightarrow ... \rightarrow
  \text{result extraction}
\end{equation*}

%% The objective of this project is to implement a regular expression
%% parser:

%% The objectives of this project is to implement an online regular
%% expression parser. A function is online if it uses a constant amount
%% of memory, not counting input and output. The parser should produce
%% full parsing, i.e. it should be able to return all groupings.

\begin{itemize}
\item Extend theory from Dub\'{e} and Feeley 
\item Develop a protocol the processes can use to communicate
\item The resulting parsing algorithm is to produce bit-values. Every
  time a decision is made during parsing, a bit-value describing this
  decision is output. The required solution can be constructed from
  these bit-values
\item The algorithm used for NFA simulation is to be based on Dub\'{e} and
  Feeleys method. It needs to be adapted so that it will be streaming
  and produce mixed bit-values
\item Develop filters (streaming processes) that can perform
  well-defined operations in the mixed bit-values without
  materializing each bit-value
\item Compare with other results and implementations
\end{itemize}

\section*{Method}
The objectives can be achieved in these steps:
\begin{enumerate}
\item Designing a protocol for communicating the mixed bit-values %% Også
  %% udvikle teori om processer
\item Implement streaming full parser. This step includes: A regular
  expression to NFA converter and a NFA and text string matcher
\item Implement streaming projections. The obvious projections being:
  A simple yes/no if the string matches the regular expression, only
  show the matching bit-values and reporting on the groupings
\item Compare run-time and memory usage of implementation with
  e.g. Perl or RE2. Perl and RE2 is chosen here for their widespread
  usage and their performance respectively
\end{enumerate}

\section*{Learning goals}
%% analyze, formalize, systematize, determine, summarize, classify,
%% assess, evaluate, quantify
At the end of this project I will be able to:

\begin{itemize}
\item Explain and use regular expressions and finite automaton theory
%\item Extend the existing algorithms and theory to limit process memory
\item Analyze algorithm by Dub\'{e} and Feeleys and evaluate memory
  consumption
\item Design new streaming algorithm based on Dub\'{e} and Feeley
\item Implement found algorithm resulting in a streaming NFA
  simulator and filters
\item Document and verify implementation
\item Discuss run-time and memory consumption of the solution, and
  compare to existing methods and implementations by complexity
  analysis and benchmarking
\end{itemize}

\bibliographystyle{plain}
\bibliography{references}


\end{document}
