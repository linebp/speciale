\begin{abstract}
    Regular expressions is a popular field. It has seen much research
    over the years and many use regular expressions as a part of their
    daily routine. The uses are widely varied and range from the
    programmer doing search and replace operations on source code to
    the biologist looking for common patterns in amino acids. This
    means there is a rich supply of regular expression engine
    implementations, some are general purpose and some are geared for
    some specific purpose.

    In this thesis we will present a design and a prototype of a
    regular expression engine. It is able to match and extract the
    values of captured groups. The design splits the process into
    several components. Our components are streaming and use constant
    memory for a fixed regular expression, with the exception of one
    non-streaming component. We also evaluate the results and compare
    our regular expression engine with other regular expression engine
    implementations.
\end{abstract}
\renewcommand{\abstractname}{Resum\'{e}}
\begin{abstract}
  Regulære udtryk er et populært område. Over årene er der blevet
  forsket meget i dette emne og mange bruger dem som en del af deres
  daglige rutine. Brugsomårederne er mangeartede, fra programmøren der
  udfører søg og erstat operationer på kildekode til biologen der
  leder efter mønstre i aminosyrer. Alt dette betyder at der er er
  rigt udvalg af forskellige implementationer af regulære udtryk,
  nogen er almengyldige og andre er mere egnede til særlige formål.

  I dette speciale vil vi præsentere et design og en prototype af en
  fortolker af regulære udtryk. Den er i stand til at genkende tekst
  strenge og udtrække værdier af grupper. Designet deler arbejdsbyrden
  op i flere enkeltkomponenter. Vores komponenter er ``streaming'' og
  bruger konstant hukommelse for et fast regulært udtryk, med
  undtagelse af en enkelt komponent. Vi vil også evaluere de opnåede
  resultater og sammenligne vores prototype med andre eksisterende
  implementationer.
\end{abstract}
