\section{Future work}
\label{sec:future}

In the following we will explain in more detail the areas that could
be interesting to continue in after this project. We have described
some extensions to the regular expression, internationalization, a few
extra filters and have a few notes on concurrency.

\subsection{Extending the current regular expression feature-set}

Regular expressions in real world usage is more complicated than what
we have described in this thesis. Here is a list with some of the
features that the regular expressions presented here could be extended
with.

\paragraph{Counted repetitions} A shorthand for matching at least $n$, but
no more than $m$ times can easily be implemented. Using industry
standard notation, the repetition $e\{3\}$ expands to $eee$,
$e\{2,5\}$ expands to $eee?e?e?$ and $e\{2,\}$ expands to $eee*$,
where $e$ is some regular expression.

\paragraph{Non-greedy or lazy quantifiers} Traditionally the quantifiers will
match as much as possible, they are greedy. Non-greedy quantifiers
will match as little as possible.

\paragraph{Character class shorthands} Often used character classes,
like \textsf{[A-Za-z0-9\_]} for a word character, often has
shorthands. It makes for shorter and more readable regular
expressions.

\paragraph{Unanchored matches} In this thesis we have assumed that the
regular expression will match the whole of the string. In practice, it
is often useful to find out if the regular expression matches a
substring of the input string. An unanchored match has implicit
non-greedy \textsf{.*} appended and prepended. Here it would also be
very useful to have the ability to restart matching where the previous
left of. 

\paragraph{More escape sequences} Special
escape sequences for characters like tab, newline, return, form feed,
alarm and escape are useful. They are more readable in a regular
expression than the actual characters themselves, because they will
show as blank space in most editors. 

\paragraph{Assertions} Assertions does not consume characters from the
input string. They assert properties about the surrounding text. Most
languages and libraries provide the start and end of line and word
boundary assertions. There are also general assertions, like the
lookahead assertion \textsf{(?=$r$)} which asserts that the text after
the current matches $r$.

\paragraph{Case insensitive matching} This could be implemented as the
poor man's version of the case insensitive match: Every character $a$
matched case insensitively is expanded to a character class
$[aA]$. This might not be the best idea performance wise, since the
character classes are so expensive to simulate.

\subsection{Internationalization}

What this long word covers over is basically integrating other
character sets than the ASCII. Internationalization also makes
character class shorthands mean different things. The word character
class from above would vary according to locale, for example in a
danish setting it would make more sense defining it as:
\textsf{[A-Åa-å0-9\_]}.


\subsection{More and better filters}

\paragraph{Copy} A copy filter can easily be implemented, just write
input to two output channels. These channels could be standard input
and error. An effect very similar can already be achieved with the
utility \texttt{tee}, this will read from standard input and write to
standard output and files.

\paragraph{Serialize} The serialization filter
dumps the contents of the captured groups with no formatting. It would
be beneficial to the user to have some kind of formatting to
distinguish the different captured groups.

\paragraph{Applying regular expressions line by line} It would not be
enough to just insert a end-of-stream character after every
new-line. We would probably have to rethink the protocol as well. We
would need some way of signaling to the programs downstream that this
is a new match, but the same regular expression should be applied. 

\subsection{Concurrency}

We can not split up neither the regular expression nor the input
string for concurrent processing. To split up the regular expression,
we would need to know exactly how much the sub-regular expressions
each consume of the input string. This we cannot know unless we
actually do the match. Splitting up the string instead would also
require knowledge we can only gain by actually performing the match,
we would need to know in what state the simulation process should
start at this particular input string symbol. 

The current setup is a pipeline. Only the streaming filters are able
to process data upon reception, the non-streaming filters gathers all
data in the input stream before beginning the processing. On most
Unix-like systems it is possible to chain programs together with
pipes. The output of a program is directly fed to the input of the
next program in the chain. This is usually implemented so that all the
programs in the chain is started at the same time. The scheduler is
then responsible for managing the processes. Although the problem is
not parallel in nature, by dividing the solution into discrete
components we can at least utilize each streaming component
simultaneously.

%% This problem is not embarrassingly parallel, but we do have some
%% potential for concurrency. Our information only flows one way and the
%% values in our streams are only dependent on the previous values, the
%% latter only applies to our streaming programs. Once a value has been
%% outputted, there is nothing stopping the next program in the pipeline
%% processing this value. Whether or not this potential is exploited is
%% dependent on the available resources and the scheduler. 



To gain more control over the concurrency we could implement the
programs as processes. This would allow us to tweak the scheduling
even further. It is however doubtful that this will give a performance
boost based on better control over the concurrency, as the scheduler
already does a good job of this. The performance boost will more
likely come from faster communication channels. 

In a Unix environment there will be no loss of data in a pipeline,
if for example a program can produce data faster than the receiving
program can read. The data will be buffered until the receiving
program is ready to read the data. If the buffer fills up, the
producer will be suspended until there again is room in the
buffer. This could mean we are buffering data twice, once in the
buffer used by the operating system between programs in a pipeline and
once in the buffer used by the programs own input and output.
