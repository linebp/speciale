\section{Conclusion}
\label{sec:conclusion}

In this thesis we have demonstrated a prototype of our design and
compared it to existing implementations of regular expression
engines. We have seen how our design is viable, both in terms of
run-time and memory consumption, both have good asymptotic bounds in
theory and in practice. The weak point in the design is the size of
the mixed bit-values output. This size determines the run-time and in
one case the memory consumption of the filters. We have only observed
a linear relationship between the size of the input string and the
size of the mixed bit-values, though it could in theory be as big as
the product of the sizes of the input string and the regular
expression.

Using the main program with the 'trace' filter we can also (in some
cases) use our framework for compression and the 'serialize' filter
for decompression. The idea is that, if you have a regular expression
matching a string, the resulting bit-values will take up less space
than the string itself. 
