\section{Conclusion}
\label{sec:conclusion}

In this thesis we have designed and demonstrated a prototype of our
design and compared it to existing implementations of regular
expression engines. We have explained the reasoning behind our design
and reasoned about theoretical performance, both in terms of run-time
and storage requirements. We then implemented the design, and
demonstrated its viability in practice, and that our expected
asymptotic bounds holds.  

We have discussed a weak point in the design regarding the size of the
mixed bit-values output which in turn determines the run-time and in
one case the memory consumption of the filters. We have only observed
a linear relationship between the size of the input string and the
size of the mixed bit-values. A worst-case analysis of the size tells
us that the size could be as big as the product of the size of the
input string and the regular expression. Practically speaking, most
simulations will however not fall into this category.

A drawback for ordinary users is the need for rewriting the regular
expression for some of the filters. This has to be done by the user
according to a rewriting function, it is a fairly straightforward
procedure, but it is a big hindrance for the design presented here to
be used in a more mainstream setting.

In addition to this, we have also discussed regular expression and
finite automatons, and conducted an analysis of Dub\'{e} and Feeleys
algorithm to adapt it to our purposes, namely to communicate progress
between filters. 

Finally, we would like to note that while this project obviously only
constitutes a prototype, the concept has some interesting potential
applications. Creative use of the 'trace' and 'serialize' filters can,
in some cases, be used for compression, to name one example. The idea
is that, if you have a regular expression matching a string, the
resulting bit-values will take up less space than the string itself.
