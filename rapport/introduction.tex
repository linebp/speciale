\section{Introduction}

This masters thesis presents a design and an implementation of a
regular expression engine with focus on memory consumption.

Regular expressions is an important tool for matching strings of
text. In many text editors and programming languages they provide a
concise and flexible way of searching and manipulating text. They find
their use in areas like data mining, spam detection, deep packet
inspection and in the analysis of protein sequences, see
\cite{pedersen2010}.


\subsection{Motivation}

Regular expressions is a popular area in computer science and has seen
much research. They are used extensively both in academia and in
business. Many programming language offer regular expressions in some
form, either as an embedded feature or as a stand alone
library. There are many different flavors of regular expression and
implementations, each adapted to some purpose. 

Many of the existing solutions gives no guarantees on their memory
consumption. In this project we will focus on a streaming solution,
that is we will, where possible, use a constant amount of memory for a
fixed regular expression. We will build a general framework to this
purpose.


%% Regular expressions has seen much research. Det er brugt mange steder
%% i erhvervslivet. Mange programmeringssprog tilbyder regulære udtryk,
%% enten indbygget eller som et seperat bibliotek. DEr findes mange
%% forskellige modeller tilpasset diverse formål. 

%% I det her projekt vil vi forsøge at at løse problemet ved at bruge
%% konstant hukommelse(fraregnet pladsen til det regulære udtryk). Mange
%% af de prodrukter der eksisterer idag giver ingen garantier på
%% hukommelsesforbruget. Vi vil forsøge at stille et generelt framework
%% op som kan løse dette


\subsection{Definitions, conventions and notation}

The empty string is denoted as $\upvarepsilon$. $\Sigma$ is used to
denote the alphabet, or set of symbols, used to write a string or a
regular expression. 

Automatons are represented as graphs, where states are nodes and
transitions are edges. The start state has an arrow starting nowhere
pointing to it. The accepting state is marked with double
circles. Edges has an attached string, indicating on which input
symbol this particular transition is allowed.

Regular expressions will be written in sans serif font:
\textsf{a\textbar b} and strings will be written slanted: \textsl{The
  cake is a lie}.


\subsection{Objectives and limitations}

The objectives of this thesis is to extend existing theory and design,
implement and evaluate a prototype. We will be extending theory by
Dub\'{e} and Feeley \cite{Dube2000} and Henglein and Nielsen
\cite{Henglein2010}. The extended theory will be used in designing a
streaming regular expression engine. The design will be implemented in
a prototype and finally we will evaluate and compare with existing
solutions.

We aim to address these topics in this thesis:

\begin{itemize}
\item Extend existing theory by Dub\'{e} and Feeley \cite{Dube2000}
  and Henglein and Nielsen \cite{Henglein2010}.

\item Create a prototype implementation

\item Compare the prototype with existing solutions 

\item Conclude and propose extensions and improvements on the work
\end{itemize}

\subsubsection{Limitations}

The focus is on designing and implementing a streaming regular
expression engine. There are many general purpose features and
optimizations that can be considered necessary in a full-fledged
regular expression engine that are only peripherally considered
here. In situations where we are faced with a choice, we have
generally favored simplicity and robustness.

%% Fokus på at bevise ideen ikke et stort featuresæt, den kan ikke serach
%% and replace

\subsection{Summary of contributions}

The main contribution of this thesis work is a streaming regular
expression engine based on Dub\'{e} and Feeley \cite{Dube2000} and
Henglein and Nielsen \cite{Henglein2010}. We present, implement and
evaluate a working prototype that demonstrates that our solution is
both technically viable and in many cases preferable from a
resource-consumption standpoint compared to existing industry
solutions.

\subsection{Thesis overview}

Section 2 gives an introduction to regular expressions and finite
automatons. In section 3 we describe the architecture of our
implementation. Section has the implementation specific details. In
section 5 we describe the behavior of the implemented prototype,
suggest some optimizations and describe how we implemented some of
them. Section 6 compares our implementation to existing
implementations. In section 7 we have the related work. Section 8
describes future work, improvements to the design and the
implementation of the prototype. Lastly we have the conclusion on our
theoretical and practical work in section 9.
