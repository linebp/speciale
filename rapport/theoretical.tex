\section{Analysis of algorithms}
\label{sec:theoretical}
In this section we analyze the complexity of the various programs and
filters. The analysis is presented as
tables of functions, with a short description of what it does and what
other functions it calls and whether or not this is done in a
loop. Most importantly we will include the complexity, for run-time and storage requirements of each function
in the tables. The analysis is a worst-case analysis - in some situations this outcome is unlikely barring a pathological input. This will be noted in the case.\todo{jan: verify and check that this is actually noted}

In the following sections $n$ denotes the length of the regular
expression and $m$ is the length of the input string.

Each component is then summarized based on the contents of the function tables.

\subsection{Constructing the NFA}

In tables \ref{tab:re2nfa1}, \ref{tab:re2nfa2} and \ref{tab:re2nfa3}
we have a overview of the functions involved in constructing a NFA. We
have the theoretical upper bound on run time and memory consumption in
big-o notation. Most functions are straightforward but a few deserve a
comment. 

\lstinline{ptrlist_patch} is the function that patches a list of
dangling pointers to a state. The pointers are only dangling once in
their lifetime and do not become dangling again once 
patched to a state. The upper bound on the total number of dangling
pointers is two times the number of states and the state count is
linear in the size of the regular expression, making the upper bound
on the number of pointers linear in the size of the regular expression
also. The total amount of work is linear in the size of the regular
expression. The number of times the \lstinline{ptrlist_patch} function
is called is linear in the size of the regular expression. The
amortized cost of calling the \lstinline{ptrlist_patch} function is
therefore O(1).

\lstinline{ptrlist_free}, see argument for \lstinline{ptrlist_patch}.

\lstinline{cc2fragment} also has a loop over the regular
expression. The reason why this does not become a O($n^2$) operation is
that the loop counter is shared between the two functions. Any
progress made by one function in the regular expression is shared with
the other.

We note that all functions called by the main loop function
\lstinline{re2nfa} are O(1) both in run time and memory, apart from
\lstinline{cc2fragment} see above comment. \lstinline{re2nfa}
allocates a stack for use in the construction process, the number of
elements in the stack is the number of characters in the regular
expression. We call the functions a number of times that are linear in
the size of the regular expression, we therefore conclude that
constructing a NFA is O($n$) in both run time and O($n$) in memory
consumption.

\ctable[
  caption = Analysis of \texttt{re2nfa} (part 1),
  label = tab:re2nfa1
]
{lp{5cm}cc}
{
  \tnote[a]{Amortized cost}
}
{\FL
  Function & Description & Run time & Memory \ML
  \lstinline{state} & Allocates memory for states & O(1) & O(1) \NN
  \lstinline{fragment} & Assigns values for a fragment & O(1) & O(1) \NN
  \lstinline{ptrlist_list1} & Allocates pointer list of one & O(1) &
  O(1) \NN
  \lstinline{ptrlist_patch} & Patches list of pointers & O(1)\tmark[a] & O(1)
  \NN
  \lstinline{ptrlist_append} & Appends two lists of pointers & O(1) &
  O(1) \NN
  \lstinline{ptrlist_free} & Frees list of pointers  &  O(1)\tmark[a]
  & O(1) \NN
  \lstinline{read_paren_type} & Determines parenthesis type & O(1) &
  O(1) \NN
  \lstinline{range} & Allocates memory for a range & O(1) & O(1) \NN
  \lstinline{parse_cc_char} & Parses a character in a character class
  & O(1) & O(1) \LL
}

\ctable[
  caption = Analysis of \texttt{re2nfa} (part 2),
  label = tab:re2nfa2
]
{lp{5cm}cc}
{
}
{\FL
  Function & Description & Run time & Memory \ML
  \lstinline{cc2fragment} & Makes a fragment of a character
  class. Calls \lstinline{state} and \lstinline{fragment} sequentially
  and \lstinline{parse_cc_range} in a loop over the regular expression
  & O(n) & O(1) \NN
  \lstinline{parse_cc_range} & Parses a range in a character
  class. Calls \lstinline{range} and \lstinline{parse_cc_char}
  sequentially  & O(1) & O(1) \NN
  \lstinline{maybe_concat} & Concatenates top two fragments. Calls
  \lstinline{ptrlist_patch} and \lstinline{ptrlist_free}
  sequentially & O(1) & O(1) \NN
  \lstinline{maybe_alternate} & Alternates top fragments. Calls
  \lstinline{state}, \lstinline{fragment}, \lstinline{ptrlist_list1} and
  \lstinline{ptrlist_append} sequentially & O(1) & O(1) \NN
  \lstinline{do_right_paren} & Does a right parenthesis. Calls
  \lstinline{maybe_concat}, \lstinline{maybe_alternate},
  \lstinline{state}, \lstinline{fragment} and
  \lstinline{ptrlist_list1} sequentially & O(1) & O(1) \NN
  \lstinline{finish_up_regex} & Finishing touches on forming the
  NFA. Calls \lstinline{state}, \lstinline{maybe_concat},
  \lstinline{maybe_alternate}, \lstinline{ptrlist_append},
  \lstinline{ptrlist_list1}, \lstinline{ptrlist_patch} and
  \lstinline{ptrlist_free} sequentially & O(1) & O(1) \NN
  \lstinline{do_quantifier} & Does quantifier. Calls
  \lstinline{state}, \lstinline{ptrlist_patch}, \lstinline{fragment},
  \lstinline{ptrlist_list1}, \lstinline{ptrlist_append}
  sequentially. & O(1) & O(1) \LL
}

\ctable[
  caption = Analysis of \texttt{re2nfa} (part 3),
  label = tab:re2nfa3
]
{lp{5cm}cc}
{
}
{\FL
  Function & Description & Run time & Memory \ML
  \lstinline{re2nfa} & Main loop. Calls \lstinline{state}, \lstinline{do_quantifier},
  \lstinline{maybe_concat}, \lstinline{maybe_alternate},
  \lstinline{fragment}, \lstinline{do_right_paren},
  \lstinline{cc2fragment} looping over the regular expression and
  \lstinline{finish_up_regex} sequentially & O(n) & O(n)
  \LL
}

\subsection{Simulating the NFA}

In table \ref{tab:match} we have the overview of functions involved in
the simulation of the NFA. We have the theoretical upper bound on run
time and memory consumption in big-O notation. Some functions deserve
a comment.

\lstinline{addstate} and \lstinline{last_addstate} marks states they
already have visited in one step. Upon encountering a state it already
has visited, it returns. This means that the total amount of work in a
single \lstinline{step} is O($n$) and not O($n^2$).

\lstinline{is_in_range} is also called once per active state in
\lstinline{step} and \lstinline{last_step}. We only create one state
per character class, what we do instead is create a list of ranges
representing the character class. The more ranges the fewer states and
vice versa. This (again) means that the total amount of work in a
single \lstinline{step} is O($n$) and not O($n^2$).

We note that for each character in the input string, in the worst case
we have to visit all states: O($n*m$). We allocate space enough for
the NFA which is O($n$) and two lists to keep track of the active
states which is also O($n$), resulting in a total of O($n$).

\ctable[
  caption = Analysis of \texttt{match},
  label = tab:match
]
{lp{5cm}cc}
{
}
{\FL
  Function & Description & Run time & Memory \ML
  \lstinline{is_in_range} & Determines if a character is accepted by a
  character class & O($n$) & O(1) \NN
  \lstinline{read_bit} & Reads character from input & O(1) & O(1) \NN
  \lstinline{write_bit} & Outputs bit & O(1) & O(1) \NN
  \lstinline{last_addstate} &  Add a state with final character in
  input string. Calls \lstinline{write_bit}
  \lstinline{last_addstate} & O($n$) & O(1) \NN 
  \lstinline{last_step} & Advance simulation final character. Calls
  \lstinline{last_addstate}, \lstinline{is_in_range} and \lstinline{write_bit} in a loop over
  active states & O($n$) & O(1) \NN
  \lstinline{addstate} & Add a state. Calls \lstinline{write_bit} and
  \lstinline{addstate} & O($n$) & O(1) \NN
  \lstinline{step} & Advances simulation one character. Calls
  \lstinline{addstate} and \lstinline{write_bit} in a loop over 
  active states & O($n$) & O($n$) \NN
  \lstinline{match} & Matches a NFA with a string & O($m*n$) & O($n$)
  \NN
  \LL
}

\subsection{The 'match' filter}
Since this filter is very simple -  we only
call \lstinline{write_bit} and \lstinline{read_bit} once per input
character - we have elected to just provide a textual summary here. This filter has run time linear in the input length and a memory consumption of O(1).\todo{jan: changed}

\subsection{The 'groupings' filter} 

In table \ref{tab:groupings} we have the overview of the functions in
the 'groupings' filter, $o$ is the input length. 

\lstinline{follow_epsilon} follows all possible transitions for a
channel. The number of possible transitions is bounded the number of
actual transitions O($n$). No loops are entered, since these requires
a character from input. It is called virtually\todo{jan: virtually?} once for every input
character. This leads us to overall complexity for this filter:
O($n*o$). The memory used is the memory for the NFA and the list of
active channels, we cannot have more active channels than states:
O($n$).

\ctable[
  caption = Analysis of the 'groupings' filter,
  label = tab:groupings
]
{lp{5cm}cc}
{
}
{\FL
  Function & Description & Run time & Memory \ML
  \lstinline{channel} & Allocates memory for channel & O(1) & O(1) \NN
  \lstinline{channel_copy} & Copies a channel & O(1) & O(1) \NN
  \lstinline{channel_free} & Frees a channel & O(1) & O(1) \NN
  \lstinline{channel_remove} & Removes a channel from the channel
  list. Calls \lstinline{channel_free}
  & O(1) & O(1) \NN
  \lstinline{follow_epsilon} & Follow all possible transitions from
  current state. Calls \lstinline{write_bit} &
  O($n$) & O(1) \NN
  \lstinline{do_end} & Channel has ended. Calls
  \lstinline{channel_remove} and \lstinline{write_bit} & O(1) & O(1) \NN
  \lstinline{do_split} & Splits the channel. Calls
  \lstinline{channel_copy} & O(1) & O(1) \NN
  \lstinline{do_one} & Channel takes the transition marked 1. Calls
  \lstinline{write_bit} and \lstinline{follow_epsilon} & O($n$) & O(1) \NN
  \lstinline{do_zero} & Channel takes the transition marked 0. Calls
  \lstinline{write_bit} and \lstinline{follow_epsilon} & O($n$) & O(1) \NN
  \lstinline{do_escape} & Handles escaped character in input. Call
  \lstinline{follow_epsilon} and \lstinline{write_bit} & O($n$) & O(1)
  \NN
  \lstinline{read_mbv} & Main loop. Calls \lstinline{read_bit},
  \lstinline{write_bit}, \lstinline{do_split}, \lstinline{do_end},
  \lstinline{do_zero}, \lstinline{do_one}, \lstinline{do_escape} in a loop over the input and
  \lstinline{follow_epsilon} & O($n*o$) & O($n$) \NN
  \lstinline{main} & Calls \lstinline{re2nfa} and \lstinline{read_mbv}
  & O($n*o$) & O($n$)
  \LL
}


\subsection{The 'trace' filter} 

In table \ref{tab:trace} we have the overview of the functions in the
trace' filter, $o$ is the input length. The version of the 'trace'
filter analyzed here, is the optimized one.

\lstinline{channel_write_bit} reallocates memory when it runs out, the
allocation strategy is to double the amount of memory every time we
run out. The complexity of copying the data over all calls to
\lstinline{channel_write_bit} is 1 + 2 + 4 + 8 + ... $o$ =
O($o$). Because this is the cost of all calls to
\lstinline{channel_write_bit}, the cost of \lstinline{trace} is only
O($o$) and not O($o^2$).

This filter is O($o$) complexity both in run time and memory. 

\ctable[
  caption = Analysis of the 'trace' filter,
  label = tab:trace
]
{lp{5cm}cc}
{
}
{\FL
  Function & Description & Run time & Memory \ML
  \lstinline{channel_write_bit} & Writes bit in channel. Reallocates
  memory if necessary & O($o$) & O($o$) \NN
  \lstinline{trace} & Reads mixed bit-values backwards and stores the
  channel containing a match. Calls \lstinline{channel_write_bit} in a
  loop over the mixed bit-values & O($o$) & O($o$) \NN
  \lstinline{read_mbv} & Reads whole input string into memory,
  reallocates memory if it runs out. Calls
  \lstinline{read_bit} in a loop over input & O($o$) & O($o$) \NN
  \lstinline{reverse} & Reverses input string & O($o$) & O($o$) \NN
  \lstinline{main} & Calls \lstinline{read_mbv}, \lstinline{reverse},
  \lstinline{trace} and \lstinline{write_bit} & O($o$) & O($o$) \NN
  \LL
}


\subsection{The 'serialize' filter}

In table \ref{tab:serialize} we have the functions in the 'serialize'
filter. $o$ is the input length. 

This is another uncomplicated filter. We basically just read input and
step through the NFA: O($o*n$). The only memory we allocate is for the
NFA: O($n$).


\ctable[
  caption = Analysis of the 'serialize filter,
  label = tab:serialize
]
{lp{5cm}cc}
{
}
{\FL
  Function & Description & Run time & Memory \ML
  \lstinline{follow} & Follows all legal transitions. Calls \lstinline{write_bit} & O($n$) & O(1)
  \NN
  \lstinline{read_bv} & Reads bit-values from input. Calls
  \lstinline{read_bit}, \lstinline{write_bit} and \lstinline{follow} in a loop over input &
  O($o*n$) & O(1) \NN
  \lstinline{main} & Calls \lstinline{re2nfa} and \lstinline{read_bv}
  & O($o*n$) & O($n$) 
  \LL
}
